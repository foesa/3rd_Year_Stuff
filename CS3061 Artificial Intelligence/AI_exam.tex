% !TEX TS-program = pdflatex
% !TEX encoding = UTF-8 Unicode

% This is a simple template for a LaTeX document using the "article" class.
% See "book", "report", "letter" for other types of document.

\documentclass[11pt]{article} % use larger type; default would be 10pt

\usepackage[utf8]{inputenc} % set input encoding (not needed with XeLaTeX)

%%% Examples of Article customizations
% These packages are optional, depending whether you want the features they provide.
% See the LaTeX Companion or other references for full information.

%%% PAGE DIMENSIONS
\usepackage{geometry} % to change the page dimensions
\geometry{a4paper} % or letterpaper (US) or a5paper or....
% \geometry{margin=2in} % for example, change the margins to 2 inches all round
% \geometry{landscape} % set up the page for landscape
%   read geometry.pdf for detailed page layout information

\usepackage{graphicx} % support the \includegraphics command and options

% \usepackage[parfill]{parskip} % Activate to begin paragraphs with an empty line rather than an indent

%%% PACKAGES
\usepackage{booktabs} % for much better looking tables
\usepackage{array} % for better arrays (eg matrices) in maths
\usepackage{paralist} % very flexible & customisable lists (eg. enumerate/itemize, etc.)
\usepackage{verbatim} % adds environment for commenting out blocks of text & for better verbatim
\usepackage{subfig} % make it possible to include more than one captioned figure/table in a single float
% These packages are all incorporated in the memoir class to one degree or another...

%%% HEADERS & FOOTERS
\usepackage{fancyhdr} % This should be set AFTER setting up the page geometry
\pagestyle{fancy} % options: empty , plain , fancy
\renewcommand{\headrulewidth}{0pt} % customise the layout...
\lhead{}\chead{}\rhead{}
\lfoot{}\cfoot{\thepage}\rfoot{}

%%% SECTION TITLE APPEARANCE
\usepackage{sectsty}
\allsectionsfont{\sffamily\mdseries\upshape} % (See the fntguide.pdf for font help)
% (This matches ConTeXt defaults)

%%% ToC (table of contents) APPEARANCE
\usepackage[nottoc,notlof,notlot]{tocbibind} % Put the bibliography in the ToC
\usepackage[titles,subfigure]{tocloft} % Alter the style of the Table of Contents
\renewcommand{\cftsecfont}{\rmfamily\mdseries\upshape}
\renewcommand{\cftsecpagefont}{\rmfamily\mdseries\upshape} % No bold!

%%% END Article customizations

%%% The "real" document content comes below...

\title{CSU33061-1 Artificial Intelligence I }
\author{Efeosa Eguavoen 17324649}
%\date{} % Activate to display a given date or no date (if empty),
         % otherwise the current date is printed 

\begin{document}
\maketitle

\section{Q2}
\subsection{(a)}
All possible transitions are defined as the set of possible resulting states generated by \(p\)
 \[P_{s_{i-1}s_i}^{a_i} = P[S_{i+1} = s_i| S_t = s_{i-1}, A_t = a_i] \], the state transition probability function,  given a state \(s_{i-1}\) and an action \(a_i\), where the probability of the resulting state is is greater than 0. The probability of all resulting states must add up to 1.

\subsection{(b)}
The transitions in (3) seem to suggest a probability function that results in 100\% probability when a corresponding action \(a_i\) is taken. This is due to the fact we don't seem to see any state more than once. As for the reward function, we can say the reward given a certain action is also constant as only one resulting state can be achieved by an action \(a_i\).
\newline Our confidence in this suggestion could increase or decrease if we were to run it multiple times and see the resulting transitions. We might see different states achieved by taking a certain action \(a_i\) which would suggest the probability function isn't 100\% given a corresponding action, also our resulting reward given an action would change also.
\newline
It's useful to consider more than one sequence or episode when trying to determine functions p and r as we can get a more accurate representation of the mappings of the functions in regards to probability and rewards as too few samples would under represent the system leading to poor decisions being made by the system due to incorrect information.
\subsection{(c \& d)}
Answer is in seperate scanned pdf  ``scanned\_answers.pdf''

\section{Q3}
\subsection{(a) - True}
The given statement is true as the learning rate in Q-learning decides if the algorithm is more willing to use previously learnt information or to discard previously known information in search of a better path. In Q-learning this is represented as a scale of 0 - 1, with 0 indicating nothing new is to be learnt as the algorithm already has the best set of information and 1 representing discarding all previously known information in favour of searching for new information and updating it as it finds it. The exploration explotation tradeoff is greatly affected by this as higher learning rates tend to explore and search for better paths to the detriment of using previously known information for the best path, while lower learning rates tend to exploit previously gained information instead of looking for new or better paths.

\subsection{(b) - True}
Prolog can be compared to a depth first search algorithm as it's a goal-orientated language that seeks unification with some declared fact. Much like dfs, it continues to search down a graph to attempt to unify with some goal until it reaches a point where it can go down no further due to some sort of error at which point it proceeds to back track and try a different path, much like using dfs on a tree or graph and reaching the max depth and then going down a different brach. In relation to this question, it would be following the logic placed in the program to the maximum depth allowed by the current path and then backtracking to try another path. But there is no guarantee maximum depth will be reached such as in hamiltonian graphs which loop back to the start deeming it bound to be incomplete.

\subsection{(c)- True}
The statement is true given the definitions of abduction and deduction. Deduction states that for states s and s', an action a can be derived as causing the transition from s to s'. Abduction states that given a state s' and an action a, it can be assumed that a state s exists that undergoes a transition via action a to become state s', hence proving abduction is the inverse of deduction.

\subsection{(d) - False}
The statement is false as values are organised based on how they correlate and correlation does not imply causation. For example drinking water and death both correlate as everyone that has drank water has died but it does not mean that drinking water is the cause of death or vice versa as the cause of death can be any number of things, proving random variables in bayes nets are not ordered based on cause and effect.

\subsection{(e) - True}
The given statement is true as naive Bayes assumptions states that every feature must be independant of every other feature given it's class, but Markov nets have the property that the future is independant of the past given the present, which is a special kind of conditional independance that's a markovian property, which may break the naive bayes assumptions. Bayes nets are directed graphs with some sort of causation between nodes while Markov nets are undirected which can caues the naive bayes classifier to be lost if moralized to a Markov net from a Bayes net.
\section{Declaration}
DECLARATION 
 \newline
 \newline
I understand that this is an individual assessment and that collaboration is not permitted. I have not received any assistance with my work for this assessment. Where I have used the published work of others, I have indicated this with appropriate citation. 
 \newline
 \newline
I have not and will not share any part of my work on this assessment, directly or indirectly, with any other student. 
 \newline
 \newline
I have read and I understand the plagiarism provisions in the General Regulations of the University Calendar for the current year, found at http://www.tcd.ie/calendar. 
 \newline
 \newline
I have also completed the Online Tutorial on avoiding plagiarism ‘Ready Steady Write’, located at http://tcdie.libguides.com/plagiarism/ready-steady-write." 
 \newline
 \newline
I understand that by returning this declaration with my work, I am agreeing with the above statement. 
 \newline
 \newline
Name: Efeosa Louis Eguavoen	
 \newline
Date:  02/05/2020
\end{document}
