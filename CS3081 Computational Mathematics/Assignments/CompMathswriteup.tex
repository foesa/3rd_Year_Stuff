% !TEX TS-program = pdflatex
% !TEX encoding = UTF-8 Unicode

% This is a simple template for a LaTeX document using the "article" class.
% See "book", "report", "letter" for other types of document.

\documentclass[11pt]{article} % use larger type; default would be 10pt

\usepackage[utf8]{inputenc} % set input encoding (not needed with XeLaTeX)
\usepackage{amsmath}
%%% Examples of Article customizations
% These packages are optional, depending whether you want the features they provide.
% See the LaTeX Companion or other references for full information.

%%% PAGE DIMENSIONS
\usepackage{geometry} % to change the page dimensions
\geometry{a4paper} % or letterpaper (US) or a5paper or....
% \geometry{margin=2in} % for example, change the margins to 2 inches all round
% \geometry{landscape} % set up the page for landscape
%   read geometry.pdf for detailed page layout information

\usepackage{graphicx} % support the \includegraphics command and options

% \usepackage[parfill]{parskip} % Activate to begin paragraphs with an empty line rather than an indent

%%% PACKAGES
\usepackage{booktabs} % for much better looking tables
\usepackage{array} % for better arrays (eg matrices) in maths
\usepackage{paralist} % very flexible & customisable lists (eg. enumerate/itemize, etc.)
\usepackage{verbatim} % adds environment for commenting out blocks of text & for better verbatim
\usepackage{subfig} % make it possible to include more than one captioned figure/table in a single float
% These packages are all incorporated in the memoir class to one degree or another...
%%% HEADERS & FOOTERS
\usepackage{fancyhdr} % This should be set AFTER setting up the page geometry
\pagestyle{fancy} % options: empty , plain , fancy
\renewcommand{\headrulewidth}{0pt} % customise the layout...
\lhead{}\chead{}\rhead{}
\lfoot{}\cfoot{\thepage}\rfoot{}

%%% SECTION TITLE APPEARANCE
\usepackage{sectsty}
\allsectionsfont{\sffamily\mdseries\upshape} % (See the fntguide.pdf for font help)
% (This matches ConTeXt defaults)

%%% ToC (table of contents) APPEARANCE
\usepackage[nottoc,notlof,notlot]{tocbibind} % Put the bibliography in the ToC
\usepackage[titles,subfigure]{tocloft} % Alter the style of the Table of Contents
\renewcommand{\cftsecfont}{\rmfamily\mdseries\upshape}
\renewcommand{\cftsecpagefont}{\rmfamily\mdseries\upshape} % No bold!

%%% END Article customizations

%%% The "real" document content comes below...

\title{Exam 2020 Computational Mathematics}
\author{Efeosa Louis Eguavoen - 17324649}
%\date{} % Activate to display a given date or no date (if empty),
         % otherwise the current date is printed 

\begin{document}
\maketitle

\section{Q4}
\begin{equation} 
	\begin{split}
P_2(x) = f(a) + f'(a)\frac{(x-a)}{1!} + f''(a)\frac{(x-a)^2}{2!} \\
 = 3-17(2)^3+-51(2)^2\frac{2.5-2}{1}+ -102(2)\frac{(2.5-2)^2}{2!}\\
 =-133-102-25.5 =  -260.5\\
 F(x) - P_2(x)  =\\
=  -262.625 + 260.5 = -2.125
	\end{split}
\end{equation}
\section{Q5}
\begin{equation}
\begin{split}
f(x) = 16x^5 - 73x^2 -133\\
x_0 = x \\ x_1 = 2.5 \\
x_{i+1} = x_i - \frac{f(x_i)(x_{i-1}-x_i)}{f(x_i-1)-f(x_i)}\\
x_{i+1} = 2.5 - \frac{f(2.5)(3-2.5)}{f(3)-f(2.5)}\\
2.270973 = 2.5 - \frac{973.25(3-2.5)}{3098-973.25}\\
x_{i+1} = 2.270973 - \frac{f(2.270973)(2.5-2.270973)}{f(2.5)-f(2.270973)}\\
2.068259 = 2.270973 - \frac{456.966852(2.5-2.270973)}{456.966852-973.25}\\
1.958756 =2.068259 - \frac{f(2.068259)(2.270973-2.068259)}{f(2.270973)-f(2.068259)}\\
1.911576 =1.958756 - \frac{f(1.958756)(2.068259-1.958756)}{f(2.068259)-f(1.958756)}\\
1.90128 =1.911576 - \frac{f(1.911576)(1.958756-1.911576)}{f(1.958756)-f(1.911576)}\\
1.900475 =1.901285 - \frac{f(1.901285)(1.911576-1.901285)}{f(1.911576)-f(1.901285)}\\
\end{split}
\end{equation}
\section{Q6}
\begin{equation}
\begin{split}
\begin{pmatrix}
25 & 5 & 4 \\
10 & 8 & 16 \\
8 & 12 & 22
\end{pmatrix} \\
U = 25 - 0 = 25 \\
L = (10 - 0) / 25 = 0.4 \\
L = (8 - 0) / 25 = 0.32\\
U = 5 - 0 = 5\\
U = 8 - (0 + ( 5 * 0.4 ) = 2) = 6\\
U = 
\begin{pmatrix}
 25& 5 &  \\
0 & 6 &  \\
0 & 0 & 
\end{pmatrix}
L=  
\begin{pmatrix}
1 & 0& 0 \\
0.4 & 1 & 0 \\
.32 &  & 1
\end{pmatrix} \\
L = (12 - (0 + ( 5 x 0.32 ) = 1.6)) / 6 = 1.73333 \\
U = 4 - 0 = 4 \\
U = 16 - (0 + ( 4 x 0.4 ) = 1.6) = 14.4 \\
U = (0 + ( 4 x 0.32 ) = 1.28) - (1.28 + ( 14.4 x 1.73333 ) = 26.2400) = -4.2400
\\
U = 
\begin{pmatrix}
 25& 5 & 4 \\
0 & 6 & 14.4 \\
0 & 0 & -4.24
\end{pmatrix}
L=  
\begin{pmatrix}
1 & 0& 0 \\
0.4 & 1 & 0 \\
.32 & 1.7333 & 1\\
\end{pmatrix} 
\end{split}
\end{equation}
\section{Q7}
\begin{equation}
\begin{split}
x_1 = \frac{(2-(7x_2 + 3x_3))}{12} \\
x_2 = \frac{(-5-(1x_1 + 1x_3))}{5} \\
x_3 = \frac{(6-(2x_1 + 7x_2))}{-11}\\ \\
1st Iteration\\
x_1 = \frac{(2-(7(3)+ 3(5)))}{12} = -\frac{17}{6}\\
x_2 = \frac{(-5-(1(-\frac{17}{6}) + 1(5)))}{5} = -\frac{43}{30}\\
x_3 = \frac{(6-(2(-\frac{17}{6}) + 7(-\frac{43}{30})))}{-11}= -\frac{217}{110}\\
2nd Iteration \\ 
x_1 = \frac{(2-(7(-\frac{43}{30})+ 3( -\frac{217}{110})))}{12} = \frac{1481}{990}\\
x_2 = \frac{(-5-(1( \frac{1481}{990}) + 1( -\frac{217}{110})))}{5} = -\frac{2239}{2475}\\
x_3 = \frac{(6-(2( \frac{1481}{990}) + 7(-\frac{2239}{2475})))}{-11}= -\frac{7766}{9075}\\
3rd Iteration \\
x_1 = \frac{(2-(7( -\frac{2239}{2475})+ 3(-\frac{7766}{9075})))}{12} = 0.90666\\
x_2 = \frac{(-5-(1( 0.90666) + 1(-\frac{7766}{9075})))}{5} = -1.0115 \\
x_3 = \frac{(6-(2( 0.90666 ) + 7(-1.0115)))}{-11}= -1.0243\\
\end{split}
\end{equation}
\section{Q8 }
\begin{equation}
\begin{split}
\begin{pmatrix}
4 & 5 \\
6 & 5
\end{pmatrix}
\begin{pmatrix}
1\\
1
\end{pmatrix} 
= 
\begin{pmatrix}
9\\11
\end{pmatrix}
=11
\begin{pmatrix}
9/11\\
1
\end{pmatrix} \\
\begin{pmatrix}
4 & 5 \\
6 & 5
\end{pmatrix}
\begin{pmatrix}
\frac{9}{11}\\
1
\end{pmatrix} 
=
\begin{pmatrix}
\frac{91}{11}\\
\frac{109}{11}
\end{pmatrix} = \frac{109}{11}
\begin{pmatrix}
\frac{91}{109}\\
1
\end{pmatrix} \\
\begin{pmatrix}
4 & 5 \\
6 & 5
\end{pmatrix}
\begin{pmatrix}
\frac{91}{109}\\
1
\end{pmatrix} 
=
\begin{pmatrix}
\frac{909}{109}\\
\frac{1091}{109}
\end{pmatrix} = \frac{1091}{109}
\begin{pmatrix}
\frac{909}{1091}\\
1
\end{pmatrix} \\
\begin{pmatrix}
4 & 5 \\
6 & 5
\end{pmatrix}
\begin{pmatrix}
\frac{9091}{1091}\\
1
\end{pmatrix} 
=
\begin{pmatrix}
\frac{9091}{1091}\\
10.00
\end{pmatrix} = 10.00
\begin{pmatrix}
.8333\\
1
\end{pmatrix} 
\end{split}
\end{equation}
\section{Q9}
\begin{equation}
\begin{split}
f(a_0) = f(2) = 4*log_2 2 = 4\\
f(a_1) = f(3) = 9*log_2 3 = 14.2646\\
f(a_2)= f(7) = 49*log_2 7 = 137.5604
\end{split}
\end{equation}
\begin{equation}
\begin{split}
(A_0,y_0) : a_0 = y_0 = 4 \\
(A_1,y_1) : \frac{y_2-y_1}{x_2-x_1} = \frac{14.2646-4}{3-2} = 10.26464 \\
(A_2,y_2): \frac{\frac{y_2-y_1}{a_2-a_1} - \frac{y_1-y_0}{a_2-a_0}}{a_2-a_0} \\
= \frac{\frac{137.5604-14.2645}{7-3} - 10.26464}{7-2} \\
= 4.111185
\end{split}
\end{equation}
\section{Q10}
\begin{equation}
\begin{split}
\int_{0}^{2\pi}\frac{1}{2+cos x}dx\\
x =\frac{1}{2}(t(b-a)+a+b) = \frac{1}{2}(t(2\pi-0)+0+2\pi)\\
\frac{2\pi t+2\pi}{2} = \pi t + \pi \\
dx = \frac{1}{2}(b-a)dt = \frac{1}{2}(2\pi)dt = \pi dt \\
\int_{-1}^{1}f(t)dt = \frac{\pi}{2+cos(\pi t+\pi)}dt \\
= C_1*f(t_1) + C_2*f(t_2)+C_3*f(t_3) \\ = 
.5555556*f(-.77459667) + .8888889*f(0) + .5555556*f(.77459667)\\
= .6324614064 + 2.792526838 + .6324614064 = 4.05745
\end{split}
\end{equation}
\end{document}
