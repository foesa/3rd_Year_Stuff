\documentclass[12pt,a4paper]{report}
\usepackage[utf8]{inputenc}
\usepackage{amsmath}
\usepackage{amsfonts}
\usepackage{amssymb}
\usepackage{listings}
\title{CSU33081 Computational Mathematics \\ Assignment 1}
\author{Efeosa Eguavoen \\ 17324649}
\usepackage{color} %red, green, blue, yellow, cyan, magenta, black, white
\definecolor{mygreen}{RGB}{28,172,0} % color values Red, Green, Blue
\definecolor{mylilas}{RGB}{170,55,241}
\lstset{breaklines = true}

\begin{document}
\maketitle
\newpage
\section{Exercise 2.31 }
Part (a):

    (i) 4 
    (ii) 13
    (iii) 26
    (iv) 18

Your Answer (i)-(iv): (ii)13

Part (b):

    (i) 0
    (ii) 12
    (iii)  7
    (iv)  4

Your Answer (i)-(iv):(i)0
\newline
\newline
Matlab code: 
\newline
\begin{lstlisting}
function val = twobytwo(matrix)
    val = (matrix(1,1)*matrix(2,2)) - (matrix (1,2)*matrix(2,1));
end

function val2 = threebythree(matrix)
    first = matrix(1,1)*twobytwo([matrix(2,2),matrix(2,3);matrix(3,2),matrix(3,3)]);
    second = matrix(1,2)* twobytwo([matrix(2,1),matrix(2,3);matrix(3,1),matrix(3,3)]);
    third = matrix(1,3)*twobytwo([matrix(2,1),matrix(2,2);matrix(3,1),matrix(3,2)]);
    val2 = (first-second)+third;
end

function val3 = fourbyfour(matrix)
    tempMat = [0,0,0,0,0,0,0,0,0];
    incr = 1;
    curAns = 0;
    for s = 1:4 
        for i = 1:4
            for j = 1:4
                if (i ~= 1 && j ~= s)
                    tempMat(incr) = matrix(i,j);
                    incr = incr +1;
                end
            end
        end
        sendMat = reshape(tempMat,[3,3]);
        if s == 1
            curAns = matrix(1,s)*threebythree(sendMat);l
        elseif mod(s,2) == 0
            curAns = curAns - (matrix(1,s)*threebythree(sendMat));
        else
            curAns = curAns + (matrix(1,s)*threebythree(sendMat));
        end
        incr = 1;
    end
    val3 = curAns
end

\end{lstlisting}
\section{Question 3.2}
Question: Determine the root of $f(x) = x - 2e^{-x}$ by: 
\begin{itemize}
\item[(a)] Using the bisection method. Start with a= 0 and  b = 1, and carry out the first three iterations.
\item[(b)] Using the secant method. Start with the two points, x1 = 0 and x2 = 1, and carry out the first three iter­ations. 
\item[(c)] Using Newton's method. Start at x1 = 1 and carry out the first three iterations. 
\end{itemize}
Part (a):
\begin{itemize}
\item[(i)] 0.1241
\item[(ii)] 0.08125
\item[(iii)] 0.074995
\item[(iv)] 0.003462
\end{itemize}
\textbf{Your Answer:}
\newline
Bisection Method: is a bracketing method for finding a numerical solution of an equation of the form $f(x) = 0$ when it is known that withing a given interval $[a,b], f(x)$ is continuous and the equation has a solution.
\newline
The algorithm for the bisection method is as follows:
\begin{itemize}
\item[1.] Choose first interval by finding points $a$ and $b$ such that a solution exists between them ($a$ and $b$ should have different signs). For us, $a$ and $b$ have been given to us as 0 and 1 respectively.
\item[2.] Calculate the first estimate of the numerical solution $x_{NS1}$ by:
\begin{center}
$x_{NS1} = \frac{(a+b)}{2}$
\end{center}
\item[3.] Determine if the solution is between a and $x_{NS1}$ or b and $x_{NS1}$. This is done by checking the sign of the product $f(a) * f(x_{NS1})$. If the result of this is less than 0, the solution is between a and $x_{NS1}$, else if the solution is greater than 0, the solution is between $x_{NS1}$ and b.
\item[4.] Select the subinterval that contains the true solution and go back to step 2. Step 2 through 4 are repeated until error bound is attained.
\end{itemize}
Since we have step 1 already done for us we will begin with step 2.
\begin{itemize}
\item[Iteration 0:] $x_{NS1} = \frac{(0+1)}{2} = 0.5$. This is our first estimate of our numerical solution. $f(0) * f(0.5) = ((0) - 2e^{-(0)}) * ((0.5) - 2e^{-(0.5)}) = -2 * -0.7130 = 1.426$. Since this is greater than 0, we know our solution is in between $x_{NS1}$ and b.
\item[Iteration 1:] $x_{NS1} = \frac{(0.5+1)}{2} = 0.75$. This is our second estimate of our numerical solution. $f(0.5) * f(0.75) = ((0.5) - 2e^{-(0.5)}) * ((0.75) - 2e^{-(0.75)}) = -0.7130 * -0.1947 = 0.1388$. Since this is greater than 0, we know our solution is in between $x_{NS1}$ and b.
\item[Iteration 2:] $x_{NS1} = \frac{(0.75+1)}{2} = 0.875$. This is our third estimate of our numerical solution. $f(0.75) * f(0.875) = ((0.75) - 2e^{-(0.75)}) * ((0.875) - 2e^{-(0.875)}) = -0.1947 * 0.04127 = -0.0080$. Since this is less than 0, we know our solution is in between a and $x_{NS1}$.
\item[Iteration 3:] $x_{NS1} = \frac{(0.75+0.875)}{2} = 0.8125$. This is our final estimate of our numerical solution. $f(0.75) * f(0.8125) = ((0.75) - 2e^{-(0.75)}) * ((0.8125) - 2e^{-(0.8125)}) = -0.1947 * -0.07499 = -0.0146$. Since this is less than 0, we know our solution is in between $x_{NS1}$ and a.
\end{itemize}
The answer we end up with is 0.8125.. or (ii)
\newline
\newline
Part (b):
\begin{itemize}
\item[(i)] 0.72481
\item[(ii)] 0.86261
\item[(iii)] 0.62849
\item[(iv)] 0.17238
\end{itemize}
\textbf{Your Answer:}
\newline
Secant Method: is a scheme for finding a  numerical solution of an equation of the form $f(x) = 0$. The method uses two points in the neigh­borhood of the solution to  determine a new estimate for the solution. Two points are used to define a straight line, and the point where the line intersects the x-axis is the new estimate for the solution. 
\newline
The equation can be generalized to an iteration formula in which a new estimate of the solution $x_{i+1}$ is determined from the previous two solutions $x_i$ and $x_{i-1}$
\begin{center}
$x_{i+1} = x_i - \frac{f(x_i)(x_{i-1}-x_i)}{f(x_{i-1})-f(x_i)}$
\end{center}
\begin{itemize}
\item[Iteration 1:] Let $x_i = b..(1)$ and $x_{i-1} = a..(0)$. We first find our next estimate of the solution by subbing into our formula.. $x_{i+1} = 1 - \frac{f(1)(0-1)}{f(0)-f(1)}$, giving us $x_{i+1} = 0.88339$. $f(0.88339) = 0.05663$.
\item[Iteration 2:] We now repeat the process for our new estimate of the solution. $x_{i+1} = 0.88339 - \frac{f(0.88339)(1-0.88339)}{f(1)-f(0.88339)}$, giving us $x_{i+1} = 0.85154$. $f(0.85154) = -0.00197$.
\item[Iteration 3:] And again.. $x_{i+1} = 0.85154 - \frac{f(0.85154)(0.88339-0.85154)}{f(0.88339)-f(0.85154)}$, giving us $x_{i+1} = 0.85261$. $f(0.85261) = 0.00000833298$.
\end{itemize}
So our answer is 0.85261 or (ii).. probably some inaccuracies due to rounding.
\newline
\newline
\newline
Part (c):
\begin{itemize}
\item[(i)] 0.65782
\item[(ii)] 0.59371
\item[(iii)] 0.45802
\item[(iv)] 0.85261
\end{itemize}
\textbf{Your Answer:}
\newline
Newton's method is a scheme for finding a numerical solution of an equation of the form $f(x) = 0 $ where $f(x)$ if continuous and differentiable and the equation is known to have a solution near a given point. The equation can be generalized for determining the "next" solution $x_{i+1}$ from the present solution $x_i$:
\begin{center}
$x_{i+1} = x_i - \frac{f(x_i)}{f^\prime(x_i)}$
\end{center}
\begin{itemize}
\item[Iteration 1:] First easiest to find out what $f^\prime(x)$ is.. $f^\prime(x) = 2e^{-x} + 1$. We know that $x_i$ = 1, so we just need to plug it into our formula to get the next solution. $x_{i+1} = 1 - \frac{f(1)}{f^\prime(1)}$ = 0.848.
\item[Iteration 2:] $x_{i+1} = 0.848 - \frac{f(0.848)}{f^\prime(0.848)}$ = 0.8433.
\item[Iteration 3:] $x_{i+1} = 0.8433 - \frac{f(0.833)}{f^\prime(0.833)}$ = 0.852. $f(0.852) = -0.0011$.
\end{itemize}
So our answer is 0.852 or (iv).
\newline\section{Question 3.2}
Question: Determine the root of $f(x) = x - 2e^{-x}$ by: 
\begin{itemize}
\item[(a)] Using the bisection method. Start with a= 0 and  b = 1, and carry out the first three iterations.
\item[(b)] Using the secant method. Start with the two points, x1 = 0 and x2 = 1, and carry out the first three iter­ations. 
\item[(c)] Using Newton's method. Start at x1 = 1 and carry out the first three iterations. 
\end{itemize}
Part (a):
\begin{itemize}
\item[(i)] 0.1241
\item[(ii)] 0.08125
\item[(iii)] 0.074995
\item[(iv)] 0.003462
\end{itemize}
\textbf{Your Answer:}
\newline
Bisection Method: is a bracketing method for finding a numerical solution of an equation of the form $f(x) = 0$ when it is known that withing a given interval $[a,b], f(x)$ is continuous and the equation has a solution.
\newline
The algorithm for the bisection method is as follows:
\begin{itemize}
\item[1.] Choose first interval by finding points $a$ and $b$ such that a solution exists between them ($a$ and $b$ should have different signs). For us, $a$ and $b$ have been given to us as 0 and 1 respectively.
\item[2.] Calculate the first estimate of the numerical solution $x_{NS1}$ by:
\begin{center}
$x_{NS1} = \frac{(a+b)}{2}$
\end{center}
\item[3.] Determine if the solution is between a and $x_{NS1}$ or b and $x_{NS1}$. This is done by checking the sign of the product $f(a) * f(x_{NS1})$. If the result of this is less than 0, the solution is between a and $x_{NS1}$, else if the solution is greater than 0, the solution is between $x_{NS1}$ and b.
\item[4.] Select the subinterval that contains the true solution and go back to step 2. Step 2 through 4 are repeated until error bound is attained.
\end{itemize}
Since we have step 1 already done for us we will begin with step 2.
\begin{itemize}
\item[Iteration 0:] $x_{NS1} = \frac{(0+1)}{2} = 0.5$. This is our first estimate of our numerical solution. $f(0) * f(0.5) = ((0) - 2e^{-(0)}) * ((0.5) - 2e^{-(0.5)}) = -2 * -0.7130 = 1.426$. Since this is greater than 0, we know our solution is in between $x_{NS1}$ and b.
\item[Iteration 1:] $x_{NS1} = \frac{(0.5+1)}{2} = 0.75$. This is our second estimate of our numerical solution. $f(0.5) * f(0.75) = ((0.5) - 2e^{-(0.5)}) * ((0.75) - 2e^{-(0.75)}) = -0.7130 * -0.1947 = 0.1388$. Since this is greater than 0, we know our solution is in between $x_{NS1}$ and b.
\item[Iteration 2:] $x_{NS1} = \frac{(0.75+1)}{2} = 0.875$. This is our third estimate of our numerical solution. $f(0.75) * f(0.875) = ((0.75) - 2e^{-(0.75)}) * ((0.875) - 2e^{-(0.875)}) = -0.1947 * 0.04127 = -0.0080$. Since this is less than 0, we know our solution is in between a and $x_{NS1}$.
\item[Iteration 3:] $x_{NS1} = \frac{(0.75+0.875)}{2} = 0.8125$. This is our final estimate of our numerical solution. $f(0.75) * f(0.8125) = ((0.75) - 2e^{-(0.75)}) * ((0.8125) - 2e^{-(0.8125)}) = -0.1947 * -0.07499 = -0.0146$. Since this is less than 0, we know our solution is in between $x_{NS1}$ and a.
\end{itemize}
The answer we end up with is 0.8125.. or (ii)
\newline
\newline
Part (b):
\begin{itemize}
\item[(i)] 0.72481
\item[(ii)] 0.86261
\item[(iii)] 0.62849
\item[(iv)] 0.17238
\end{itemize}
\textbf{Your Answer:}
\newline
Secant Method: is a scheme for finding a  numerical solution of an equation of the form $f(x) = 0$. The method uses two points in the neigh­borhood of the solution to  determine a new estimate for the solution. Two points are used to define a straight line, and the point where the line intersects the x-axis is the new estimate for the solution. 
\newline
The equation can be generalized to an iteration formula in which a new estimate of the solution $x_{i+1}$ is determined from the previous two solutions $x_i$ and $x_{i-1}$
\begin{center}
$x_{i+1} = x_i - \frac{f(x_i)(x_{i-1}-x_i)}{f(x_{i-1})-f(x_i)}$
\end{center}
\begin{itemize}
\item[Iteration 1:] Let $x_i = b..(1)$ and $x_{i-1} = a..(0)$. We first find our next estimate of the solution by subbing into our formula.. $x_{i+1} = 1 - \frac{f(1)(0-1)}{f(0)-f(1)}$, giving us $x_{i+1} = 0.88339$. $f(0.88339) = 0.05663$.
\item[Iteration 2:] We now repeat the process for our new estimate of the solution. $x_{i+1} = 0.88339 - \frac{f(0.88339)(1-0.88339)}{f(1)-f(0.88339)}$, giving us $x_{i+1} = 0.85154$. $f(0.85154) = -0.00197$.
\item[Iteration 3:] And again.. $x_{i+1} = 0.85154 - \frac{f(0.85154)(0.88339-0.85154)}{f(0.88339)-f(0.85154)}$, giving us $x_{i+1} = 0.85261$. $f(0.85261) = 0.00000833298$.
\end{itemize}
So our answer is 0.85261 or (ii).. probably some inaccuracies due to rounding.
\newline
\newline
\newline
Part (c):
\begin{itemize}
\item[(i)] 0.65782
\item[(ii)] 0.59371
\item[(iii)] 0.45802
\item[(iv)] 0.85261
\end{itemize}
\textbf{Your Answer:}
\newline
Newton's method is a scheme for finding a numerical solution of an equation of the form $f(x) = 0 $ where $f(x)$ if continuous and differentiable and the equation is known to have a solution near a given point. The equation can be generalized for determining the "next" solution $x_{i+1}$ from the present solution $x_i$:
\begin{center}
$x_{i+1} = x_i - \frac{f(x_i)}{f^\prime(x_i)}$
\end{center}
\begin{itemize}
\item[Iteration 1:] First easiest to find out what $f^\prime(x)$ is.. $f^\prime(x) = 2e^{-x} + 1$. We know that $x_i$ = 1, so we just need to plug it into our formula to get the next solution. $x_{i+1} = 1 - \frac{f(1)}{f^\prime(1)}$ = 0.848.
\item[Iteration 2:] $x_{i+1} = 0.848 - \frac{f(0.848)}{f^\prime(0.848)}$ = 0.8433.
\item[Iteration 3:] $x_{i+1} = 0.8433 - \frac{f(0.833)}{f^\prime(0.833)}$ = 0.852. $f(0.852) = -0.0011$.
\end{itemize}
So our answer is 0.852 or (iv).
\newline

\section{Exercise 4.24}
Q 4.24

    (i) Inverse(a)=

	-0.7143 	0.0 		1.4286
0.2571 		0.1000 		0.2857
-0.2286 	-0.2000 	0.8571


	Inverse(b)=

1.6667 		2.8889 		-2.2222 	1.0000
0.0 			0.3333 		-0.3333 	0.0
-0.3333 		-0.4444 	0.1111 		0.0
1.5000 		2.0000 		-1.5000 	0.5000


(ii)

Inverse(a)=

	0.7243		 0.0 		1.3286
1.2571 		0.1000 		0.2757
-0.2386 	-0.2010 	0.9571


	 Inverse(b)=

1.6677 		2.9889  		3.2222 		1.01700
 0.3433 		-0.3433 	0.3333		0.00371
-0.3433 		-0.2879 	0.2111 		0.0	
1.2400 		2.0120 		-1.5783 	0.5600


(iii)

Inverse(a)=

	0.7143		 0.003 		2.3276
1.2671 		0.1100 		0.3759
-0.2486 	-0.2110 	0.9771


Inverse(b)=

1.6877 		3.9789  		3.2002 		2.01800
 0.3533 		-0.4433 	0.3333		0.02371
-0.3443 		-0.2999 	0.3121 		0.0382	
1.2420 		3.0130 		-1.5733 	0.5610


(iv)

Inverse(a)=

	0.8343		 1.01 		1.3336
2.2572 		0.1003 		0.3857
-0.2486 	-0.2110 	0.9671


Inverse(b)=

1.6777 		4.9889  		3.2232 		1.11700
 0.3443 		-0.3443 	0.3233		0.07371
-0.3443 		-0.2979 	0.3211 		0.07800	
1.2480 		2.1220 		-1.5883 	0.5621


Your Answer (i)-(iv):
The answer I got was (i)
\begin{lstlisting}
	function Ainv = Inverse (A)
    [n, m]=size(A); 
    if n ~= m 
        Ainv ='The matrix must be square';
        return
    end
    if n == 0 
        Ainv ='Matrix cant be empty';
        return
    end
    Ainv = eye(n);
    for r = 1 : n
        for c = r : n
            if A(c,r) ~= 0
                t = 1/A(r,r);
                for i = 1 : n
                    A(r,i) = t * A(r,i);
                    Ainv(r,i) = t * Ainv(r,i); 
                end
                for i = 1 : n
                    if i ~= r 
                        t = -A(i,r);
                        for j = 1 : n
                            A(i,j) = A(i,j) + t * A(r,j);
                            Ainv(i,j) = Ainv(i,j) + t * Ainv(r,j);
                        end
                    end
                end
            end
            break
        end
    end
end
\end{lstlisting}
\end{document}